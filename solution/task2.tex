\documentclass{report}
\usepackage[cp1251]{inputenc}
\usepackage[russian]{babel}
\usepackage{amsmath, amssymb, latexsym}
\usepackage{mathrsfs}
\usepackage{fancybox,fancyhdr}

\makeatletter
\renewcommand{\section}{\@startsection
	{section}%
	{2}%
	{0mm}%
	{-\baselineskip}%
	{0.5\baselineskip}%
	{\normalfont\large\bfseries}}
\makeatother

\setcounter{page}{139}

\setcounter{section}{7}
\newcounter{nrazd}
\setcounter{nrazd}{0}
\newcommand{\razd}[1]{\addtocounter{nrazd}{1}
	\setcounter{equation}{0}
	\textbf{\thenrazd. #1}}

\setcounter{equation}{-2}
\renewcommand{\theequation}{\thenrazd.\arabic{equation}}


\renewcommand{\thesection}{\S\ \arabic{section}.}
\newcommand{\bm}[1]{\mbox{\boldmath{\(#1\)}}}


\begin{document}
	\razd{Определяющие уравнения.} Для координат оператора, допускаемого
	системой (1.1), также вводятся индивидуальные обозначения, а именно

	\begin{equation} \label{eq2.1}
		\zeta\cdot\partial=\xi^a \partial_a + \eta^i \partial_{u^i} + \sigma\partial_p + \tau\partial_\rho,
	\end{equation}

	\noindent где все $\xi^a, \eta^i, \sigma, \tau$ есть искомые отображения $\bm{X \times Y\rightarrow R}$.

	Применение изложенного в 5.8 алгоритма перехода $E \rightarrow DE$
	никаких принципиальных затруднений не встречает. Так как
	(1.1) есть система первого порядка, то требуется лишь первое
	продолжение оператора (2.1). Переход на многообразие Е осуществляется
	просто исключением величин $u^i_0$, $p_0$, $\rho_0$ с помощью
	\noindent (1.1), причем в полученных условиях инвариантности все оставшиеся
	координаты точки продолженного пространства $u^i_k, \ p_k, \ \rho_k  $ являются
	«свободными» параметрами. Так как уравнения (1.1) квазилинейны, то полное число
	$N_n$ определяющих уравнений, получаемых после расщепления, может быть вычислено по формуле
	5(8.1). Подсчет дает $N_1,=30,\ N_2=180$ и $N_3=680$.

	Расщепление условий инвариантности относительно «свободных» параметров приводит к следующим результатам.
	Обращение в нуль совокупности квадратичных слагаемых равносильно выполнению равенств

	\begin{equation} \label{eq2.2}
		\partial_\rho \xi^\alpha = \partial_p \xi^\alpha = \partial_{u^i} \xi^\alpha = 0
		\ \ (\alpha = 0, 1, \ldots, n; i=1,\ldots,n).
	\end{equation}

	\noindent Обращение в нуль совокупности слагаемых первой степени равносильно выполнению равенств

	\begin{equation} \label{eq2.3}
		\partial_i \xi^\alpha = 0, \ \partial_1 \xi^1 = \partial_2 \xi^2 = \ldots = \partial_n \xi^n = \lambda, \ \partial_k \xi^i + \partial_i \xi^k = 0;
	\end{equation}

	\begin{equation} \label{eq2.4}
		\eta^i = u^k \partial_k \xi^i - u^i \partial_0 \xi^0 + \partial_0 \xi^i;
	\end{equation}

	\begin{equation} \label{eq2.5}
		\partial_\rho \sigma = \partial_u i \sigma = 0, \ \partial_p \tau = 0, \ \tau = \rho(\partial_p \sigma + 2\partial_0 \xi^0 - 2\lambda);
	\end{equation}

	\begin{equation} \label{eq2.6}
		\sigma\partial_p A + \tau\partial_\rho A = A \partial_p \sigma,
	\end{equation}

	где введена вспомогательная функция $\lambda : X \rightarrowtail R$.
	Наконец, оставшиеся слагаемые с нулевой степенью «свободных» параметров образуют систему

	\begin{equation} \label{eq2.1} \left.
		\begin{array}{r}
			\partial_0 \eta^i + u^k \partial_k \eta^i + \rho^{-1} \partial_i \sigma = 0, \\
			\partial_0 \tau + u^k \partial_k \tau + \rho \partial_k \eta^k = 0, \\
			\partial_0 \sigma + u^k \partial_k \sigma + A\partial_k \eta^k = 0.
		\end{array}
		\right\} \end{equation}


	Уравенения (2.2)$\--$(2.7)  образуют полную систему определяющих уравнений $DE(A)$.


	\razd{Ядро основых алгебр Ли.}
	Если функция $A$ вполне произвольна, то уравнение (2.6) может быть удволетрворено только значениями
	$\sigma = \tau = 0$. В этом случае из (2.5) следует, что $\lambda = \partial_0 \xi^0$, а уравнения (2.7) своядтся к следующим:

	\begin{equation} \label{eq3.1}
		\partial_0 \eta^i + u^k \partial_k \eta^i = 0, \ \ \partial_k \eta^k = 0.
	\end{equation}

	Подстановка в эти уравнения выражения (2.4) и последующиее расщепление полученных равенств относительно
	«свободными» в силу (2.2) и (2.4) параметров $u^k$ приводит к системе уравнений, содержащих только координаты $\xi^i$

	\begin{equation} \label{eq3.2}
		\partial_k \partial_j \xi^i = 0, \ \ 2\partial_k \partial_0 \xi^i = \delta^i_k \partial^2_0 \xi^0, \ \ \partial^2_0 \xi^i = 0.
	\end{equation}

	Эти уравнения показывают, что все производные всторого порядка от координат $\xi^i$ равны нулю и, кроме того,
	$\partial^2_0 \xi^0 = 0$. Поэтому общее решение строится без каких-либо затруднений и оказывается таким.


\end{document} %