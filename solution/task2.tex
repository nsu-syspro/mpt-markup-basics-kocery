\documentclass{report}
\usepackage[cp1251]{inputenc}
\usepackage[russian]{babel}
\usepackage{amsmath, amssymb, latexsym}
\usepackage{mathrsfs}
\usepackage{fancybox,fancyhdr}

\makeatletter
\renewcommand{\section}{\@startsection
{section}%
{2}%
{0mm}%
{-\baselineskip}%
{0.5\baselineskip}%
{\normalfont\large\bfseries}}
\makeatother

\setcounter{page}{139}

\setcounter{section}{7}
\newcounter{nrazd}
\setcounter{nrazd}{0}
\newcommand{\razd}[1]{\addtocounter{nrazd}{1}
\setcounter{equation}{0}
\textbf{\thenrazd. #1}}

\setcounter{equation}{-2}
\renewcommand{\theequation}{\thenrazd.\arabic{equation}}


\renewcommand{\thesection}{\S\ \arabic{section}.}
\newcommand{\bm}[1]{\mbox{\boldmath{\(#1\)}}}

\newtheorem{lemma}{\hspace{\parindent}Лемма}
\newtheorem{lemman}{\hspace{\parindent}Лемма}
\renewcommand{\thelemma}{\!\!.}
\renewcommand{\thelemman}{\arabic{lemman}.}

\newtheorem{theorema}{\hspace{\parindent}Теорема}
\newtheorem{theoreman}{\hspace{\parindent}Теорема}
\renewcommand{\thetheorema}{\!\!.}
\renewcommand{\thetheoreman}{\arabic{theoreman}.}
\fancyhead[R]{Это простой пример верхнего колонтитула}
\begin{document}
\fancyhead[L]{\thepage}
\fancyhead[C]{\footnotesize ГЛ. III. ОСНОВНЫЕ ГРУППЫ КОНКРЕТНЫХ СИСТЕМ УРАВНЕНИЙ}
\fancyhead[R]{Это простой пример верхнего колонтитула}

Здесь должен быть Ваш текст. Ниже примеры,
которые могут Вам потребоваться. Потом удалить!

Для заголовка параграфа использовать команду:

\section{Обыкновенные дифференциальные уравнения}
В строке \verb"\setcounter{section}{7}"
вместо <<7>> поставить число на единицу меньшее номера параграфа
из Вашего фрагмента текста.

Для заголовка раздела использовать команду:

\razd{Система уравнений первого порядка.}\\
В строке \verb"\setcounter{nrazd}{0}"
вместо <<0>> поставить число на единицу меньшее первого раздела
Вашего фрагмента текста.

Если первое нумерованное уравнение встречается раньше заголовка
раздела, то перед строкой\\
\verb"\renewcommand{\theequation}{\thenrazd.\arabic{equation}}"\\
вставить строку
\verb"\setcounter{equation}{n}",
где n на единицу меньше номера уравнения.

Индекс под символом \(\underset{1}{y}\) и индекс над символом  \(\overset{*}{t}\).

Полужирный и наклонный в математических формулах: \(\bm{X}\), \bm{R}.

Символы неравенств: \(\leqslant\), \(\geqslant\).

Угловые скобки: \(\langle x \rangle\).

Рукописное L: \(\mathscr{L}\).

\begin{lemma}
Формулировка леммы.
\end{lemma}
\begin{lemman}
Формулировка леммы.
\end{lemman}

\begin{theorema}
Формулировка теоремы.
\end{theorema}

\begin{theoreman}
Формулировка теоремы.
\end{theoreman}

Нумерованная формула
$$%\begin{equation} \label{eq1.1}
\underset{1}{y}=f(x,y),
$$%\end{equation}
и ссылка на эту формулу (\ref{eq1.1}).

Многострочное уравнение с четным числом строк и одним номером
(уравнения Коши-Римана):

%$$\begin{equation}
$$\begin{array}{rcl}
u_x &=& -v_y,\\[3pt]
v_x &=& u_y.
\end{array}
$$%\end{equation}


\newpage
\noindent условию $\partial_p S \neq 0$, в силу которого можно ввести другую задаваемую функцию $ A: \bm{R^2 \rightarrow R}$, определяемую формулой

$$ \bm{A} = -\rho\partial_p S / \partial_p S.$$

В случае политропного газа эта функция равна $A = \gamma$$p$, где постоянная $\gamma$ называется показателем адиабаты. С помощью функции $A$
последнему уравнению системы можно придать равносильную
форму

$$ d_t p + A \ div \ u = 0. $$

Рассматриваемая система имеет тип $E(n+1, n+2, 1, n+2)$.
В дальнейшем она изучается для любого натурального $n$, что
не привносит каких-либо дополнительных усложнений по сравнению с «физическими» случаями $n = 1, 2, 3$. Вычисления ведутся
в индивидуальных обозначениях согласно представлениям $\bm{X = R^{n+1}}$ $(t, x^1, \ldots, x^n)$ и $\bm{Y = R^{n+2}}$$ (u^1,\ldots,u^*,p,\rho)$. При этом опера-
тор дифференцирования отображений $\bf{X \rightarrow Y}$ записывается в виде
$\partial_x = (\partial_1, \ldots, \partial_n)$, где $\partial_0 = \partial_t$ и «материальная производная»
дается формулой $\partial_0+u^i \partial_j$. Кроме того, первые производные обозначаются сокращенно в соответствии с равенствами

$$\partial_a u^i = u^i_a, \ \ \partial_a p=p_a, \ \ \partial_a \rho=\rho_a   \ \ \ (a=0,1,\ldots,n). $$

Ниже используется соглашение о том, что латинские индексы
$i,j,k, \ldots$ принимают значения $1, \ldots, n,$ а греческие индексы
$\alpha,\beta,\ldots$ — значения $0, 1, \ldots, n$. В этих обозначениях исследуемая система имеет вид

\begin{equation} \label{eq1.1} \left.
  \begin{array}{r}
    u^i_0 +u^ku^i_k+\rho^-1 p_i = 0, \\
    \rho_0 + u^k \rho_k + \rho u^k_k = 0, \\
                p_0 + u^k p_k + Au^k_k = 0. \\
  \end{array}
\right\} \end{equation}

Для системы (1.1), обозначаемой также символом $E(A)$, решается задача групповой классификации по отношению к произвольному элементу $A$ как функции
$(p, \rho) \rightarrow A (p, \rho)$ в предположении, что $A \neq 0$.

\razd{Определяющие уравнения.} Для координат оператора, допускаемого
системой (1.1), также вводятся индивидуальные обозначения, а именно

\begin{equation} \label{eq2.1}
\zeta\cdot\partial=\xi^a \partial_a + \eta^i \partial_{u^i} + \sigma\partial_p + \tau\partial_\rho,
\end{equation}

\noindent где все $\xi^a, \eta^i, \sigma, \tau$ суть искомые отображения $\bm{X \times Y\rightarrow R}$.

Применение изложенного в 5.8 алгоритма перехода $E \rightarrow DE$
никаких принципиальных затруднений не встречает. Так как
(1.1) есть система первого порядка, то требуется лишь первое
продолжение оператора (2.1). Переход на многообразие Е осуществляется просто исключением величин $u^i_0$, $p_0$, $\rho_0$ с помощью
\newpage
\noindent (1.1), причем в полученных условиях инвариантности все остав-
шиеся координаты точки продолженного пространства $u^i_k, \ p_k, \ \rho_k  $ являются «свободными» параметрами. Так как уравнения (1.1)
квазилинейны, то полное число $N_n$ определяющих уравнений, получаемых после расщепления, может быть вычислено по формуле
5(8.1). Подсчет дает $N_1,=30,\ N_2=180$ и $N_3=680$.

Расщепление условий инвариантности относительно «свободных» параметров приводит к следующим результатам. Обращение в нуль совокупности квадратичных слагаемых равносильно
выполнению равенств

\begin{equation} \label{eq2.2}
        \partial_\rho \xi^\alpha = \partial_p \xi^\alpha = \partial_{u^i} \xi^\alpha = 0
        \ \ (\alpha = 0, 1, \ldots, n; i=1,\ldots,n).
\end{equation}

\noindent Обращение в нуль совокупности слагаемых первой степени равносильно выполнению равенств

\begin{equation} \label{eq2.3}
        \partial_i \xi^\alpha = 0, \ \partial_1 \xi^1 = \partial_2 \xi^2 = \ldots = \partial_n \xi^n = \lambda, \ \partial_k \xi^i + \partial_i \xi^k = 0;
\end{equation}

\begin{equation} \label{eq2.4}
        \eta^i = u^k \partial_k \xi^i - u^i \partial_0 \xi^0 + \partial_0 \xi^i;
\end{equation}

\begin{equation} \label{eq2.5}
        \partial_\rho \sigma = \partial_u i \sigma = 0, \ \partial_p \tau = 0, \ \tau = \rho(\partial_p \sigma + 2\partial_0 \xi^0 - 2\lambda);
\end{equation}

\begin{equation} \label{eq2.6}
        \sigma\partial_p A + \tau\partial_\rho A = A \partial_p \sigma,
\end{equation}

где введена вспомогательная функция $\lambda : X \rightarrowtail R$. Наконец, оставшиеся слагаемые с нулевао степенью «свободных» параметров образуют систему

\begin{equation} \label{eq2.1} \left.
\begin{array}{r}
        \partial_0 \eta^i + u^k \partial_k \eta^i + \rho^{-1} \partial_i \sigma = 0, \\
        \partial_0 \tau + u^k \partial_k \tau + \rho \partial_k \eta^k = 0, \\
        \partial_0 \sigma + u^k \partial_k \sigma + A\partial_k \eta^k = 0.
\end{array}
\right\} \end{equation}


Уравенения (2.2)$\--$(2.7)  образуют полную систему определяющих уравнений $DE(A)$.


\razd{Ядро основых алгебр Ли.}
Если функция $A$ вполне произвольна, то уравнение (2.6) может быть удволетрворено только значениями $\sigma = \tau = 0$. В этом случае из (2.5) следует, что $\lambda = \partial_0 \xi^0$, а уравнения (2.7) своядтся к следующим:

\begin{equation} \label{eq3.1}
        \partial_0 \eta^i + u^k \partial_k \eta^i = 0, \ \ \partial_k \eta^k = 0.
\end{equation}

Подстановка в эти уравнения выражения (2.4) и последующиее расщепление полученных равенств относительно «свободными» в силу (2.2) и (2.4) параметрво $u^k$ приводит у системе уравнений, содержащих только координаты $\xi^i$

\begin{equation} \label{eq3.2}
        \partial_k \partial_j \xi^i = 0, \ \ 2\partial_k \partial_0 \xi^i = \delta^i_k \partial^2_0 \xi^0, \ \ \partial^2_0 \xi^i = 0.
\end{equation}

Эти уравнения показывают, что все производные всторого порядка от координат $\xi^i$ равны нулю и, кроме того, $\partial^2_0 \xi^0 = 0$. Поэтому общее решение строится бех каких-либо затруднений и оказывается таким.



\end{document} %